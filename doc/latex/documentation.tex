\documentclass[a4paper, 11pt]{article}
\usepackage[left=14mm, top=23mm, text={183mm, 252mm}]{geometry} % Rozměry stránky
\usepackage[utf8]{inputenc}                                     % Kódování zdrojového souboru
\usepackage[T1]{fontenc}                                        % Kódování výstupního souboru
\usepackage[czech]{babel}                                       % Delení slov, typografická
\usepackage{lmodern}                                            % Font
\usepackage{amsthm, amssymb, amsmath}                           % Matematické výrazy
\usepackage{hyperref}

% Autor: Petr Plíhal (xpliha02@stud.fit.vutbr.cz)

\begin{document}
    
\begin{titlepage}
    \begin{center}
        \textsc{
            {\Huge 
                Vysoké učení technické v~Brně \\ \vspace{0.5em}
            }
            {\huge
                Fakulta informačních technologií
            }
        }
        \vspace{\stretch{0.381966}}\\
        {\LARGE
            IMP -- WiFi metronom \\  \vspace{0.6em}
            
        }
        
        \vspace{\stretch{0.618034}}
    \end{center}

    {\Large
        \today
        \hfill
        Petr Plíhal (xpliha02)
    }
\end{titlepage}
\tableofcontents
\newpage

\section{Úvod}
Tento dokument popisuje projekt metronomu na mikrokontroléru ESP32 do předmětu IMP. Projekt implementuje metronom s webovým rozhraním pro ovládání základních parametrů, jako jsou tempo a hlasitost.

\section{Funkce}
Metronom nabízí následující funkce:
\begin{itemize}
    \item Připojení k WiFi síti nebo vytvoření vlastního přístupového bodu \cite{platformio_board}.
    \item Webový server pro ovládání metronomu \cite{esp32_manual}.
    \item Nastavení BPM (beats per minute) v rozsahu 40--240 \cite{esp32_datasheet}.
    \item Nastavení hlasitosti bzučáku (0--255, na stránce prezentováno jako procenta).
    \item Možnost spuštění a zastavení metronomu \cite{d1_r32_manual}.
\end{itemize}

\section{Použitý hardware}
\begin{itemize}
    \item ESP32 \cite{esp32_datasheet}
    \item Bzučák a LED připojené k pinu 23
\end{itemize}

\section{Použité knihovny a vývojové prostředí}
Projekt byl implementován použitím rozšíření PlatformIO pro Visual Studio Code \cite{platformio_board}.

Projekt využívá následující knihovny:
\begin{itemize}
    \item Arduino
    \item WiFi
    \item WebServer
    \item driver/timer (pro práci s časovačem)
\end{itemize}

\section{Nastavení a spuštění}
\begin{enumerate}
    \item Zapojte bzučák (nebo LED) k pinu 23 (případně změňte pin v souboru \texttt{src/config.h}).
    \item V souboru \texttt{src/config.cpp} nastavte:
    \begin{itemize}
        \item Způsob připojení: \texttt{wifi\_as\_access\_point}
        \item Jméno a heslo přístupového bodu: \texttt{ap\_ssid}, \texttt{ap\_password}
        \item Alternativně vyplňte jméno a heslo pro připojení k WiFi síti: \texttt{wifi\_ssid}, \texttt{wifi\_password}
        \item Port webového serveru: \texttt{WebServer server(XX)}
    \end{itemize}
    \item Nahrajte kód na ESP32.
    \item Připojte se k WiFi síti nebo vytvořenému přístupovému bodu.
    \item V konzoli se zobrazí IP adresa, na které je dostupné webové rozhraní.
    \item Otevřete webový prohlížeč a zadejte IP adresu.
\end{enumerate}

\section{Používání metronomu}
\begin{itemize}
    \item Spuštění a zastavení metronomu pomocí tlačítka \texttt{Start/Stop}.
    \item Nastavení hlasitosti a BPM pomocí posuvných lišt nebo textových polí.
    \item Změny na posuvné liště se projeví ihned.
    \item Změny zadané do textového pole je třeba potvrdit tlačítkem \texttt{Nastav}.
\end{itemize}

\section{Struktura zdrojových souborů}
\begin{itemize}
    \item \texttt{src/main.cpp} -- hlavní smyčka a volání funkcí pro obsluhu webového serveru.
    \item \texttt{src/config.h}, \texttt{src/config.cpp} -- konfigurační soubory pro nastavení WiFi, pinu bzučáku a portu webového serveru.
    \item \texttt{src/timer.h}, \texttt{src/timer.cpp} -- obsluha časovače pro generování tónu.
    \item \texttt{src/web\_server.h}, \texttt{src/web\_server.cpp} -- obsluha webového serveru, zpracování požadavků a volání funkcí.
    \item \texttt{src/web\_pages.h}, \texttt{src/web\_pages.cpp} -- HTML stránka pro ovládání metronomu.
\end{itemize}

\section{Známé problémy}
\begin{itemize}
    \item Při změně BPM a hlasitosti se může stát, že se tón zastaví a znovu spustí až po uplynutí jednoho taktu.
\end{itemize}

\section{Možná vylepšení}
\begin{itemize}
    \item Přidání možnosti změny tónu bzučáku (BPM také ovlivňuje tón).
    \item Přidání \uv{presets} pro různé takty (např. 4/4, 3/4, 6/8).
    \item Zvýraznění prvního taktu hlasitějším tónem.
\end{itemize}

\section{Zdroje}
Zde uvádím odkazy na stránky a dokumenty které jsem použil během implementace. V případě použití konkrétních úseků kódu jsem zdroje zaznačil odkazem v daných zdrojových souborech.

\bibliographystyle{czechiso}
\bibliography{references}

\end{document}
